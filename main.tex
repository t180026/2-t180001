f\documentclass{article}
\usepackage[utf8]{inputenc}
\title{A4数式ビッシリ}
\author{t180001 }
\date{April 2020}

\begin{document}

\maketitle

\section{大学の入試問題に出てくる関数の問題}
はじめに、関数には一次関数、二次関数、三次関数、....とあり、高校卒業程度の知識が求められる大学入試においては一般的には三次関数までとされていることが多い。
一次関数 $y=x$ という様な初歩的な関数 比例や反比例などと一緒に学習する
二次関数 $y=x^2$  という様な曲線を表す関数 解が2つある場合もあり、解の個数を判別する判別式も同時に習うことが多い。
三次関数 $y=x^3$という様な山と谷両方がある様な曲線を表す関数 解が3つあることもある。二次関数と同様に判別式は存在するが、二次関数のそれに比べてとても複雑なので高校では紹介程度に習うことがある。 

過去に龍谷大学の入試に出題された問題には以下のものがある。     $(4x^2+9y^2-36)(4x^2-27y)>0$が表す領域を座標平面上に図示しなさい。
   同様に近畿大学の入試問題は                      $C:x^2+y^2=13$         $   C'x^2+y^2-8x+14y+13$2つの円の2つの共通接線の式とそれらの交点を求めよ。という様な問題である。
   関数の問題は難易度的には同じ様な問題でも少し着目点を変えると難易度や求められるテクニックが変わってくる。
   これらの関数を用いた微分積分の問題に関しても同様である。 龍谷大学の積分問題は$p(x)={f(x)}^2f'(x)$とおく。ただし、$f(2)=2,f(4)=5,f'(4)=7,f''(4)=-1$である時2〜4の範囲での$p(x)dx$の値を求めなさい。また微分可能な関数$g(x)$が$g(f(x))=x$を満たす。この時$g'(f(x))f'(x)$は定数となる。この値を求めなさい。くわえて、この時2〜4の範囲で$(1/g'(f(x)))dx$のあたいを求めなさいというものである。          
数学の入試問題は大学によって傾向が違うことは当たり前だが問題自体の関数はそれほど違いはない。しかし、例えば$(x^2+24x-y)(x^2+y)=0$という関数と$(x^2+24x-y)(x^2-y)=0$符号一つ違うだけで全く別の問題になったりする。数学の問題は他の科目に比べて受験者の応用力を試すことに適していることがわかる。 
 
\end{document}
